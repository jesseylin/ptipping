% Inbuilt themes in beamer
\documentclass{beamer}

% Title page details:
\title{Basin Instability}
\author{Jesse Young Lin}
\date{\today}

\begin{document}

% Title page frame
\begin{frame}
	\titlepage
\end{frame}

% % Outline frame
\begin{frame}{Outline}
	\tableofcontents
\end{frame}

% Lists frame
\section{Basins of attraction}
\section{Basin instability for fixed points}
\section{Basin instability for limit cycles}
\section{Basin instability in the RMA model}

\begin{frame}{Attractors}
	Consider the system
	\begin{align*}
		\dot x = f(x;p).
	\end{align*}
	with an associated flow $\varphi_{t}$.
	\vspace{0.5cm}
	\begin{definition}[Attractor]
		An \emph{attractor} is an invariant set of points $\Gamma(p)$ depending on the parameters
		$p$ such that for some initial condition $x_{0}$
		\begin{align*}
			\inf_{\gamma \in \Gamma(p)} \lvert \varphi_{t}(x_{0}) - \gamma \rvert \to 0
		\end{align*}
		as $t \to \infty$.
	\end{definition}

\end{frame}

\begin{frame}{Basins of attraction}
	We have an attractor $\Gamma(p)$ if we find some initial condition $x_{0}$
	that limits to it. Thus, we define
	\vspace{0.5cm}
	\begin{definition}[Basin of attraction]
		\begin{align*}
			B(\gamma, p) = \{x_{0} : \varphi_{t}(x_{0}) \to \Gamma(p)\ \textrm{as}\ t \to \infty\}
		\end{align*}
	\end{definition}
	with the distance metric defined before.
\end{frame}

\begin{frame}{An example basin of attraction}
\begin{figure}[ht]
  \centering
  \includegraphics[]{figures/path.pdf}
  \caption{\label{fig:label} }
\end{figure}

\end{frame}

\end{document}
